\section{} % Part 1
Let's start with the main function.
By the looks of it, I'd wager it performs some glut-related initialization based on whatever arguments the main function is given.
\texttt{glutInitDisplayMode} probably does some display initialization with \texttt{GLUT\_RGBA} bitwised OR'd with \texttt{GLUT\_SINGLE} as its argument.

The next three calls configure the window and set its initial position, size and title.
The title-setting call's main function is possibly creating the window -- actually making it appear -- but with the argument passed it probably has the side-effect of setting the window's title.
Or the argument might be for something completely different.

The draw function is set as the drawing function for the display and then the main loop of glut is set in motion.

The draw function draws a black background.

\subsection{Setup}
I did this on a Macbook Pro Retina 2013 (\texttt{MacBookPro11,1}) running \texttt{OS X 10.9.4 (13E28)} ``Mavericks''.
\begin{itemize}
    \item \texttt{freeglut-2.8.1}
        \begin{itemize}
            \item installed using homebrew
        \end{itemize}
    \item \texttt{gcc-4.8.3}
    \item GNU Make 3.81
\end{itemize}

Vim was used for editing.
The Makefile was copied from lazyfoo's OpenGL tutorial (the mac-with-terminal version)\footnote{\url{http://lazyfoo.net/tutorials/OpenGL/01_hello_opengl/mac/cli/index.php}}, with some alterations of my own.
