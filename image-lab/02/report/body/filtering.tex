\section{Filtering}
\subsection{Something something discrete 1D convolutions}

\subsubsection{Compute some discrete convolution}
The given ``16 pixel wide image`` is actually 17 pixels wide.
$6+5+6 = 17$.
It's okay.
Don't sweat it.

Table~\ref{tab:11a} was produced using the partial filter method for the two first and last pixels in the image.

\begin{table}[H]
\centering
\begin{tabular}{|c|c|c|c|c|c|c|c|c|c|c|c|c|c|c|c|c|}
    \hline
    0 & 0 & 0 & 0 & 1 & 2 & 3 & 4 & 5 & 4 & 3 & 2 & 1 & 0 & 0 & 0 & 0 \\
    \hline
\end{tabular}
\caption{The discrete convolution of the given 5px wide filter and 17px wide image.}
\label{tab:11a}
\end{table}


\subsubsection{Do I see the need for any padding in the example?}
Nope!
If we pad the edges with zeroes the result is the same.
Padding with anything other than zeroes changes the result in a way that isn't representative of the original image with the given filter applied.

\subsection{Convolution}

\subsubsection{Compute the continuous convolution of two identical rectangular functions.}
The functions are $1$ in the range $[-\frac{1}{2}, \frac{1}{2}]$ and $0$ otherwise.

$$
(f * g)(t) =
\begin{cases}
    1+t & \text{if } -1 < t \leq 0 \\
    1-t & \text{if } 0 < t \leq 1 \\
    0 & \text{otherwise}
\end{cases}
$$

As $g$ slides along the axis from left to right it hits $f$ when $t=-1$.
For $t \in (-1, 0]$ the intersection of the areas of $f$ and $g$ grow from 0 to 1.
For $t \in (0, 1]$ the intersection decreases from 1 to 0.
The area of the intersection is 1 when $t$ is 0.

\subsubsection{What shape did I expect?}
It would make more sense to ask me this \textit{before} I computed the convolution.

But whatever: triangle!

