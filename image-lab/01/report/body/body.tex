\usemintedstyle{tango}
\section{Matlab}
I didn't install Matlab.
Couldn't figure out how to do it on my linux workstation.
Yes I tried following Orakeltjenesten's guide, but when I copied the files over the permissions were reset so I couldn't run the installer.
Sure I could've reconfigured the permissions myself, but installing Octave seemed like less of a hassle at that point.

\section{Exercise Book}
This is my exercise book.
I've no idea how many like it exist.
But as I have previously stated, this one is mine.

\section{Basic Image Manipulation}
\subsubsection{}

\inputminted[linenos=true]{octave}{../code/exercise03.m}

\subsubsection{} % task 2
That's a really good question oh boy I am speculating so hard right now.

We normalize the flat-field image.
And $\frac{x}{y} \geq x$ when $y \in (0, 1]$.
Which means the pixel values in the original image are increased.
Had we used multiplication they would have decreased, resulting in an overall darker image.

Flat-field correction is used to remove artifacts caused by variations in the detector's sensitivity and/or distortions in the optical path.
The intensity of light is reduced as it passes through objects.
The idea is to correct for this, which means the result should be an image with a higher intensity.

\section{Point Processing}

\subsection{Intensity Transform}

\inputminted[linenos=true]{octave}{../code/exercise04-1.m}

Could not observe any differences between image generated by line 8 and 12.

\subsection{Histogram Equalization}
\subsubsection{}
If there's no spikes in either end of the histogram.

\subsubsection{}

\section{Kahoot!}

Normalization means:
\begin{enumerate}[(a)]
    \item to make normal
    \item to convert values in a range [0, x] to [0, 1]
    \item to convert values in a range [x, y] to [y, x]
    \item the same as orthogonalization
\end{enumerate}


